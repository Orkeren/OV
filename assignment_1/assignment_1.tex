
% Document class: article with font size 11pt
% ---------------
\documentclass[11pt,a4paper]{article}

\setlength{\textwidth}{165mm}
\setlength{\textheight}{240mm}
\setlength{\parindent}{0mm} % S{\aa} meget rykkes ind efter afsnit
\setlength{\parskip}{\baselineskip}
\setlength{\headheight}{0mm}
\setlength{\headsep}{0mm}
\setlength{\hoffset}{-2.5mm}
\setlength{\voffset}{0mm}
\setlength{\footskip}{15mm}
\setlength{\oddsidemargin}{0mm}
\setlength{\topmargin}{0mm}
\setlength{\evensidemargin}{0mm}

\usepackage[a4paper, hmargin={2.8cm, 2.8cm}, vmargin={2.5cm, 2.5cm}]{geometry}
\usepackage[super]{nth}
\PassOptionsToPackage{hyphens}{url}\usepackage{hyperref}
\usepackage{eso-pic} % \AddToShipoutPicture
\usepackage{float} % This will allow precise picture placement, use [H].


% Call packages
% ---------------
\usepackage{comment} %Possible to comment larger sections
%http://get-software.net/macros/latex/contrib/comment/comment.pdf
\usepackage[T1]{fontenc} %oriented to output, that is, what fonts to use for printing characters.
\usepackage[utf8]{inputenc} %allows the user to input accented characters directly from the keyboard

%Support Windows TeXStudio
\usepackage[T1]{fontenc}
\usepackage{lmodern}

%http://mirrors.dotsrc.org/ctan/fonts/fourier-GUT/doc/latex/fourier/fourier-doc-en.pdf
\usepackage[english]{babel}														     % Danish
\usepackage[protrusion=true,expansion=true]{microtype}				                 % Better typography
%http://www.khirevich.com/latex/microtype/
\usepackage{amsmath,amsfonts,amsthm, amssymb}							 % Math packages
\usepackage[pdftex]{graphicx} %puts to pdf and graphic
%http://www.kwasan.kyoto-u.ac.jp/solarb6/usinggraphicx.pdf
\usepackage{xcolor,colortbl}
%http://mirrors.dotsrc.org/ctan/macros/latex/contrib/xcolor/xcolor.pdf
%http://texdoc.net/texmf-dist/doc/latex/colortbl/colortbl.pdf
\usepackage{tikz} %documentation http://www.ctan.org/pkg/pgf
\usepackage{parskip} %http://www.ctan.org/pkg/parskip
%http://tex.stackexchange.com/questions/51722/how-to-properly-code-a-tex-file-or-at-least-avoid-badness-10000
%Never use \\ but instead press "enter" twice. See second website for more info

% MATH -------------------------------------------------------------------
\newcommand{\Real}{\mathbb R}
\newcommand{\Complex}{\mathbb C}
\newcommand{\Field}{\mathbb F}
\newcommand{\RPlus}{[0,\infty)}
%
\newcommand{\norm}[1]{\left\Vert#1\right\Vert}
\newcommand{\essnorm}[1]{\norm{#1}_{\text{\rm\normalshape ess}}}
\newcommand{\abs}[1]{\left\vert#1\right\vert}
\newcommand{\set}[1]{\left\{#1\right\}}
\newcommand{\seq}[1]{\left<#1\right>}
\newcommand{\eps}{\varepsilon}
\newcommand{\To}{\longrightarrow}
\newcommand{\RE}{\operatorname{Re}}
\newcommand{\IM}{\operatorname{Im}}
\newcommand{\Poly}{{\cal{P}}(E)}
\newcommand{\EssD}{{\cal{D}}}
% THEOREMS ----------------------------------------------------------------
\theoremstyle{plain}
\newtheorem{thm}{Theorem}[section]
\newtheorem{cor}[thm]{Corollary}
\newtheorem{lem}[thm]{Lemma}
\newtheorem{prop}[thm]{Proposition}
%
\theoremstyle{definition}
\newtheorem{defn}{Definition}[section]
%
\theoremstyle{remark}
\newtheorem{rem}{Remark}[section]
%
\numberwithin{equation}{section}
\renewcommand{\theequation}{\thesection.\arabic{equation}}


\author{
  \Large{
    Frenzel, Sven Uhrenholdt (\href{mailto:sven@frenzel.dk}{sven@frenzel.dk}) - 130793 - cdn769}\\
}
\title{
  \huge{Compilers\\}
\vspace{3cm}
\Large{\nth{1} Weekly Assignment}
}

\begin{document}

\AddToShipoutPicture*{\put(0,0){\includegraphics*[viewport=0 0 700 600]{include/natbio-farve}}}
\AddToShipoutPicture*{\put(0,602){\includegraphics*[viewport=0 600 700 1600]{include/natbio-farve}}}

\AddToShipoutPicture*{\put(0,0){\includegraphics*{include/nat-en}}}

\clearpage\maketitle
\thispagestyle{empty}

\clearpage\newpage
\thispagestyle{plain}


\section*{Task 1}
I am a second year Computer Science bachelor student.

I am somewhere in the middle wrt. functional programming and SML, but not very experienced assembler programming as I haven't taken the course ARK. I have seen that Thorkil has posted on Piazza what he beliefs to be the necessary MIPS curriculum for this course.

I do not have any specific expectations except for a deeper understanding of what a programming language is.

\section*{Task 2}
If call-by-value is used the function returns the value $13$ for \textit{r}, $3$ for \textit{x} and $1$ for \textit{y}.

When using call-by-value all modifications to variables only happen in the scope of the function and not in the scope of the caller. Thus the modifications to \textit{a}, \textit{b} and \textit{c} in the function \textit{f} don't affect \textit{x} and \textit{y} in the scope of the main function.

If call-by-name is used the function returns the value $19$ for \textit{r}, $8$ for \textit{x} and $3$ for \textit{y}.

When using call-by-name every mention of the formal parameters is replaced with the actual parameters. This means that the callee can write into the scope of the caller and overwrite it's own parameters. In the first line of \textit{f} the value of \textit{x} in the caller's and callee's scope is set to $4$ $(3+1)$. In the second line the value of \textit{y} is set to 3. In the third line the value of \textit{x} is set to 8.
Thus the return value is $8+3+8=19$.


If call-by-value-result is used the function returns the value $13$ for \textit{r}, $7$ for \textit{x} and $2$ for \textit{y}. 

This happens as the function is executed as call-by-value and then updates the actual parameters in the caller's scope just before returning.

\section*{Task 3}
With static scoping both f(3) and f(5) would print 4. This happens because g() is called without any argument and thus uses the global x.
Any x declared or used in h(int y) and f(int x) is bound to the scope of the function and thus not accessible to g().

With dynamic scoping f(3) would print 3 and f(5) would print 7. In f(3) the use of x as formal argument acts as declaration of x and as g() is called before f(3) terminates the most recent declaration of x is 3.

In f(5) x first is declared in f(int x), but then declared again in h(5) as 7. Then g() is called from h(5) and the most recent declaration of x is 7.


\section*{Task 4}
See attached .sml file.

Thanks to your comments I managed to clean up the code a \textbf{lot} and make it run without any errors or warnings. Furthermore compr is implemented.
\end{document}
